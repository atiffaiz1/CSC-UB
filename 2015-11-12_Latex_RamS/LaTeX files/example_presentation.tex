\documentclass[xcolor=svgnames]{beamer}
\usetheme{theme1}
\usepackage[backend=bibtex,bibstyle=numeric-comp]{biblatex}
\newcommand{\aheader}[2]{\action<#1->{#2}}
% first argument: slide number to appear from, second argument: content of header 
\newcommand{\hiddencell}[2]{\action<#1->{#2}}
% first argument: slide number to appear from, second argument: content of cell
\newcommand{\putCitation}[2]{\footnotetext[#1]{\tiny{\citeauthor{#2} \citefield{#2}{shortjournal} \textbf{\citefield{#2}{volume}}, \citefield{#2}{pages} (\citeyear{#2})}}}

\addbibresource{references.bib}

\title{A Simple Presentation}
\author{Ramachandran Subramanian}
\institute[UB]{
    Organized by: Computational Sciences Club (CSC)\\
    Sponsored by: Graduate Student Association (GSA)
}
\date{November 13, 2015}

\begin{document}
    {
        \setbeamertemplate{footline}{} 
        \begin{frame}
            \titlepage
        \end{frame}
    }
    \addtocounter{framenumber}{-1}
    \begin{frame}
        \frametitle{Welcome to beamer}
        \begin{center}
            Hurray! Our first presentation!!!
        \end{center}
    \end{frame}
    \begin{frame}
        \frametitle{Overlays}
        \begin{itemize}
            \item Pause
            \item Action
            \item Onslide
            \item Only
            \item ...
        \end{itemize}
    \end{frame}
    \begin{frame}
        \frametitle{Pause}
        One way of making \LaTeX~ wait is by using $\backslash$pause.
        \begin{itemize}
            \item No pause here.
            \item Pause is included here. Beamer will wait for me to press a key or click the mouse. \pause
            \item No pause here.\\
        \end{itemize}
        Writing $\backslash$pause after each bullet point is clearly not efficient. So we use the $[<+->]$ option along with itemize: $\backslash$begin\{itemize\}$[<+->]$.
        \begin{itemize}[<+->]
            \item Pause after each bullet point.
            \item Pause after each bullet point.
            \item Pause after each bullet point.
        \end{itemize}
    \end{frame}
    \frame{
        \frametitle{The power of overlays\footnote{\href{http://tex.stackexchange.com/questions/24499/}{http://tex.stackexchange.com/questions/24499/}}}
        Here's my table:
        \begin{tabular}{rrr}
        \aheader{2}{Column A} & \aheader{3}{Column B} & \aheader{4}{Column C} \\
        \hiddencell{2}{1} &     \hiddencell{3}{2} &     \hiddencell{4}{3} \\
        \hiddencell{2}{4} &     \hiddencell{3}{5} &     \hiddencell{4}{6} \\
        \hiddencell{2}{7} &     \hiddencell{3}{8} &     \hiddencell{4}{9}
        \end{tabular}        
    }
    \begin{frame}
        \frametitle{References in presentations}
        To cite references, I use the custom made $\backslash$putCitation command:
        From Ref\footnotemark[2], we know that
        \putCitation{2}{Lepage1972}
    \end{frame}
\end{document}
